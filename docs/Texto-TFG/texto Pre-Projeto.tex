Proposta de Trabalho de Conclusão de Curso
Título: [Sistema de Interação Humano-Computador para processamento da língua natural]
Nome do Estudante: [Pedro Guilherme Gabriel Maurer]
Orientador: [Alexandre de Oliveira Zamberlan]
Objetivo:
O presente trabalho tem como objetivo principal projetar, implementar e avaliar um sistema de interação humano-computador que processe a língua natural baseado em um Grande modelo de Linguagem (LLM), configurando, integrando funcionalidades de Fala para texto (STT) para inferência, Texto para Fala (TTS) para resposta auditiva, porém tudo offline.
Justificativa:
A relevância deste trabalho reside na crescente demanda por sistemas de interação humano-computador mais eficientes e adaptáveis, especialmente aqueles que lidam com processamento de linguagem natural. Com a proliferação de assistentes virtuais e tecnologias de IA, a capacidade de entender e processar a linguagem humana de forma precisa e eficiente tornou-se essencial em uma variedade de contextos, desde assistência ao cliente até automação de tarefas. Este projeto visa preencher uma lacuna importante, oferecendo uma solução que não depende de conexão com a internet, garantindo assim privacidade e segurança dos dados dos usuários, além de proporcionar um desempenho mais estável em ambientes onde a conectividade pode ser limitada.
Referências:
Brown, T. B., Mann, B., Ryder, N., Subbiah, M., Kaplan, J., Dhariwal, P., ... & Amodei, D. (2020). Language Models are Few-Shot Learners. arXiv preprint arXiv:2005.14165.
Devlin, J., Chang, M. W., Lee, K., & Toutanova, K. (2018). BERT: Pre-training of Deep Bidirectional Transformers for Language Understanding. arXiv preprint arXiv:1810.04805.
Vaswani, A., Shazeer, N., Parmar, N., Uszkoreit, J., Jones, L., Gomez, A. N., ... & Polosukhin, I. (2017). Attention is all you need. In Advances in neural information processing systems (pp. 5998-6008).
Atividade	Março	Abril	Maio	Junho	Julho	Agosto	Setembro	Outubro	Novembro	Dezembro
Pesquisar sobre PLN	✔	✔	✔	 	 	 	 	 	 	 
Escrever sobre PLN	 	✔	✔	 	 	 	 	 	 	 
Pesquisar e testar bibliotecas PLN	 	 	✔	✔	✔	 	 	 	 	 
Escrever sobre bibliotecas PLN	 	 	 	✔	✔	✔	 	 	 	 
Pesquisar, testar e escrever sobre biblioteca PLN	 	 	 	 	 	✔	✔	✔	 	 
Modelar aspectos funcionais e estruturais do sistema de interface	 	 	 	 	 	 	✔	✔	✔	 
Protótipar interfaces de comunicação	 	 	 	 	 	 	 	✔	✔	✔
Escrever TFG1	 	 	 	✔	 	 	 	 	 	 
Apresentar TFG1	 	 	 	✔	 	 	 	 	 	 
Implementar módulos de comunicação com processamento da língua natural	 	 	 	 	 	 	 	 	✔	✔
Integrar os módulos construídos no sistema de interação	 	 	 	 	 	 	 	 	 	✔
Projetar e implementar um estudo de caso para o processo de avaliação do sistema proposto	 	 	 	 	 	 	 	 	 	✔
Escrever TFG2	 	 	 	 	 	 	 	 	 	✔
Apresentar TFG2	 	 	 	 	 	 	 	 	 	✔

