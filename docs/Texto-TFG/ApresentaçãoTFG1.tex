\documentclass{beamer}

% Choose how your presentation looks.
% For more themes, color themes, and font themes, see:
% http://deic.uab.es/~iblanes/beamer_gallery/index_by_theme.html

\mode<presentation>
{
  \usetheme{Madrid}      % or try Darmstadt, Madrid, Warsaw, ...
  \usecolortheme{crane}  % or try albatross, beaver, crane, seagull...
  \usefonttheme{serif}   % or try serif, structurebold, ...
  \setbeamertemplate{navigation symbols}{}
  \setbeamertemplate{caption}[numbered]
}

\usepackage{ragged2e}    % Pacote para justificação
\justifying
\centering               % Centraliza o texto justificado

\usepackage[brazil]{babel}
\usepackage[utf8x]{inputenc}
\usepackage{xcolor}
\usepackage{listings}
\lstset
{
    language=[LaTeX]TeX,
    breaklines=true,
    basicstyle=\tt\scriptsize,
    keywordstyle=\color{blue},
    identifierstyle=\color{magenta},
}
    \title[TFG Ciência da Computação]{\includegraphics[height=2.1cm]{images/logoUFN.png}\\Protótipo de Interação Humano-Computador\\ para Processamento da Língua Natural em LLMs}
    \author{Aluno:Pedro Guilherme Gabriel Maurer \\ Orientador: Alexandre Zamberlan}
    \date{UFN - Universidade Franciscana}

    \logo{\includegraphics[height=1cm]{images/logo_cc.png}}

\AtBeginSection[]
{
  \begin{frame}<beamer>
    \frametitle{Agenda}
    \tableofcontents[currentsection,currentsubsection]
  \end{frame}
}

\begin{document}

\begin{frame}
  \titlepage
\end{frame}

\section{Introdução}
\begin{frame}[allowframebreaks]{Introdução}
  \begin{block}{Contexto}
    \begin{itemize}
      \item IHC
      \item PLN
      \item Chatbots
      \item Grandes Modelos de Linguagem (LLM)
      \item Transformers
    \end{itemize}
  \end{block}
\end{frame}

\begin{frame}{Justificativa}
  \begin{block}{}
    \begin{itemize}
      \item Demanda por sistemas de interação humano-computador mais eficientes e adaptáveis com PLN
      \item Proliferação de assistentes virtuais e tecnologias de IA
      \item Assistentes \textit{offline}: segurança e privacidade
    \end{itemize}
  \end{block}
\end{frame}

\begin{frame}[allowframebreaks]{Objetivos}
  \begin{block}{Geral}
    Projetar, implementar e avaliar um protótipo de interação humano-computador que processe a língua natural baseado em um Grande Modelo de Linguagem (LLM), configurando e integrando funcionalidades de Fala para Texto (STT) por meio de \textit{prompts}, Texto para Fala (TTS) para resposta auditiva, porém \textit{offline}.
  \end{block}
  
  \begin{block}{Específicos}
    \begin{itemize}
      \item Explorar tecnologias de PLN e LLM;
      \item Pesquisar e testar bibliotecas para PLN;
      \item Pesquisar e testar bibliotecas Python de PLN (Transformers);
      \item Pesquisar e testar ambientes de validação ou avaliação do sistema construído.
    \end{itemize}
  \end{block}
\end{frame}

\section{Revisão bibliográfica}
\begin{frame}[allowframebreaks]{Revisão bibliográfica}
  \begin{itemize}
    \item IHC (\textit{Interação Humano-Computador})
    \item GPT (\textit{Generative Pre-trained Transformer})
    \item PLN (Processamento da Língua Natural ou \textit{Natural Language Processing} (NLP))
    \item \textit{Prompts} e Chatbots
    \item Python e bibliotecas:
      \begin{itemize}
        \item Transformers
        \item Hugging Face Hub
        \item Sentence Transformers
        \item PyTorch e TorchVision
        \item Tokenizers
      \end{itemize}
  \end{itemize}
\end{frame}
%%%%%%%%%%%%%%%%%%%%%%%%%%%%%%%%%%%%%%%%%%%%%%%%%%%%%%%%%%%%%%%%%%%%%%%%%%%%%%%%%%%%

\section{Interação Humano-Computador}

\begin{frame}{Interação Humano-Computador}
    \begin{itemize}
        \item IHC é o estudo da interface entre usuários e sistemas computacionais.
        \item Objetivo: criar sistemas eficientes e agradáveis para os usuários.
        \item Boas práticas: usabilidade, acessibilidade, eficiência e satisfação do usuário \cite{Norman13}.
    \end{itemize}
\end{frame}

\section{Processamento da Língua Natural}
\begin{frame}
    \frametitle{Processamento da Língua Natural}
    \framesubtitle{Definição e Tarefas}
    \begin{block}{Definição}
        PLN é uma subárea da IA focada na interação entre computadores e linguagem humana, compreendendo, interpretando e gerando linguagem de forma semelhante aos humanos \cite{hapke2019natural}.
    \end{block}
    \begin{itemize}
        \item Compreensão de texto
        \item Tradução automática
        \item Resumo automático
        \item Análise de sentimentos
        \item Geração de texto
        \item Classificação de texto
    \end{itemize}
\end{frame}


\begin{frame}{IHC e Chatbots com Tecnologia GPT}
    \begin{itemize}
        \item Influência da IHC na qualidade da interação com chatbots:
        \begin{itemize}
            \item Design de interface amigável e intuitiva.
            \item Feedback do usuário para melhorias contínuas.
            \item Personalização da experiência do usuário.
            \item Segurança de dados na utilização dos chatbots.
        \end{itemize}
    \end{itemize}
\end{frame}

\begin{frame}{Tecnologia GPT}
    \begin{itemize}
        \item \textit{Generative}: geração de novos textos.
        \item \textit{Pré-treinado}: aprendizado com grande quantidade de dados, ajustável para tarefas específicas \cite{devlin2018bert}.
        \item \textit{Transformer}: rede neural que é a peça-chave desta tecnologia \cite{Attention-Is-All-You-Need}.
    \end{itemize}
\end{frame}

\subsection{Transformers}

\begin{frame}{Transformers: Visão Geral}
    \begin{itemize}
        \item Introduzido pela Google em 2017 para tradução de texto \cite{Attention-Is-All-You-Need}.
        \item Modelo: recebe trechos de texto e gera novo texto prevendo palavras ou tokens subsequentes enriquecendo o texto com contribuições refinadas do modelo preditivo\cite{holtzman2019curious}.
    \end{itemize}
\end{frame}

\begin{frame}{Conceito de \textit{Corpus}}
    \begin{itemize}
        \item \textit{Corpus}: conjunto de textos utilizado para treinar ou avaliar modelos de linguagem.
        \item Fundamental para a interação e previsões dos modelos de linguagem \cite{Foundations-of-Statistical-Natural-Language-Processing}.
    \end{itemize}
\end{frame}

\begin{frame}{Fluxo do Transformer e Arquitetura dos modelos de fundação}
    \begin{figure}[!h]
        \centering
        \includegraphics[width=2in]{images/The Transformer - model architecture.png}
        \caption{Fluxo do Transformer e modelo de arquitetura \cite{Attention-Is-All-You-Need}.}
        \label{fig:arquiteturaTransformer}
    \end{figure}
\end{frame}

\begin{frame}{Geração de Texto com Transformers}
    \begin{itemize}
        \item Fragmentação da entrada em \textit{tokens}. 
        \item Tokens mapeados para vetores numéricos que codificam seu significado.
        \item Mecanismo de atenção atualiza os significados dos tokens.
    \end{itemize}
\end{frame}

\begin{frame}{Rede Neural Multicamada Perceptron \textit{Feedforward}}
    \begin{itemize}
        \item Vetores processados por uma Rede Neural Multicamada Perceptron.
        \item Operações executadas como uma série de multiplicação de matrizes.
    \end{itemize}
\end{frame}

\begin{frame}[allowframebreaks]{Visualização da Multiplicação de Matrizes}
    \begin{figure}[!h]
        \centering
        \includegraphics[width=4.3in]{images/Transformer inference in tokens flow.png}
        \caption{Visualização do fluxo de multiplicação entre matrizes durante geração.}
        \label{fig:VisualizacaoMatrizes.}
    \end{figure}
\end{frame}

\begin{frame}[allowframebreaks]{LLMs (Grandes Modelos de Linguagem)}
  \begin{itemize}
    \item Grandes volumes de texto utilizados para treinamento.
    \item Exemplos: ChatGPT da OpenAI, llama da Meta, gemini da google...
    \item Possibilidade de Refinamento e ajuste para tarefas específicas.
    \item Capacidade de gerar textos coerentes e contextualmente apropriados.
  \end{itemize}
\end{frame}

\begin{frame}{O que modelos Baseados em Transformers podem Fazer}
    \begin{figure}[!h]
        \centering
        \includegraphics[width=2in]{images/Transformer-apps-672x459.jpg}
        \caption{Tipos de entradas dos modelos e suas saídas\cite{nvidea}.}
        \label{fig:Tipos de entradas dos modelos e suas saídas}
    \end{figure}
\end{frame}

\section{Prompts em LLM Chatbots}
\begin{frame}
    \frametitle{Prompts em LLM Chatbots}
    \framesubtitle{Definição e Utilização}
    \begin{block}{Definição}
        Entradas de texto fornecidas ao modelo para iniciar ou direcionar a geração de texto \cite{NEURIPS2020_1457c0d6}.
    \end{block}
    \begin{itemize}
        \item Iniciar interações
        \item Direcionar a geração de texto
        \item Solicitar informações específicas
        \item Personalizar a interação
        \item Controlar o tom e o estilo da resposta
    \end{itemize}
\end{frame}

\section{Outras Tecnologias LLM}
\begin{frame}
    \frametitle{Modelos Pré-treinados Multimodais}
    \begin{block}{Definição}
        Modelos treinados em dados de várias modalidades (texto, imagem, áudio) \cite{radford2021learning}.
    \end{block}
    \begin{itemize}
        \item Compreensão e geração multimodal
        \item Aplicações em visão computacional, reconhecimento de voz, etc.
    \end{itemize}
\end{frame}

\begin{frame}
    \frametitle{Pensamentos dos Agentes em LLMs}
    \begin{itemize}
        \item Entrada de dados
        \item Compreensão
        \item Geração de resposta
        \item Revisão
        \item Decisão
    \end{itemize}
    \begin{block}{Processo}
        A rede neural realiza operações para gerar uma resposta coerente e relevante \cite{NEURIPS2020_1457c0d6}.
    \end{block}
\end{frame}

\begin{frame}
    \frametitle{Retrieval Augmented Generation (RAG)}
    \begin{block}{Definição}
        Modelos que combinam recuperação de dados externos com geração de texto \cite{lewis2020retrieval}.
    \end{block}
    \begin{itemize}
        \item Exemplo de uso: clima atual, informações de documentos
    \end{itemize}
\end{frame}
%%%%%%%%%%%%%%%%%%%%%%%%%%%%%%%%%%%%%%%%%%%%%%%%%%%%%%%%%%%%%%%%%%%%%%%%%%%%%%%%%%%%
\begin{frame}[allowframebreaks]{Trabalhos relacionados}
    
    \begin{itemize}
    \item \textit{Text-Generation-WebUI} \cite{text-generation-webui}
    \item \textit{Lobe Chat} \cite{Lobe-Chat}
    \item \textit{Open WebUI}\cite{open-webui}
\end{itemize}    


\end{frame}

\begin{frame}{Text-Generation-WebUI}
    \begin{figure}[!h]
    \centering
    \includegraphics[width=4.0in]{images/Text generation web UI.png}
    \caption{Interface do Text-Generation-WebUI.} 
    \label{fig:Text-Generation-WebUI}
    \end{figure}
\end{frame}

\begin{frame}{Lobe Chat}
    \begin{figure}[!h]
    \centering
    \includegraphics[width=4.0in]{images/Lobe HUB UI.png}
    \caption{Interface do Lobe Chat.} 
    \label{fig:Lobe Chat}
    \end{figure}
\end{frame}

\begin{frame}{Open WebUI}
    \begin{figure}[!h]
    \centering
    \includegraphics[width=4.0in]{images/OpenWebUi.png}
    \caption{Interface do Open WebUI.} 
    \label{fig:Open WebUI}
    \end{figure}
\end{frame}

\section{Proposta}  
\begin{frame}{Materias e métodos}

\begin{block}{Metodologia}
    A metodologia usada SCRUM mais KANBAN. Além disso,  uma interface deve ser construída para processar a língua natural e avaliar a proposta.    
\end{block}


As ferramentas utilizadas são:
\begin{itemize}
    \item Linguagem Python;
    \item Bibliotecas do universo Python: transformer, NLTK, langchain-community, ollama, faster-whisper, pyttsx3, SpeechRecognition;
    \item Modelo pré-treinado LLAMA3-8B;
    \item Visual Studio Code;
    \item Astah;
    \item Trello.
\end{itemize}
\end{frame}

\begin{frame}{Visão Geral}
    \begin{figure}[!h]
    \centering
    \includegraphics[width=4.0in]{images/Mapa mental.png}
    \caption{Mapa Mental com as áreas da pesquisa.} 
    \label{fig:Mapamental}
    \end{figure}
\end{frame}

\begin{frame}[allowframebreaks]{Estudo de caso}

\begin{figure}[!h]
\centering
\includegraphics[width=3in]{images/UseCase.png}
\caption{Caso de uso do protótipo.} 
\label{fig:Casodeuso}
\end{figure}

\end{frame}

\begin{frame}[allowframebreaks]{Protótipo}

\begin{figure}[!h]
\centering
\includegraphics[width=3in]{images/Prototipo test.png}
\caption{Primeiros Testes do protótipo.} 
A proposta do protótipo de interação visa abordar a comunicação tanto escrita quanto falada, estabelecendo uma interação bidirecional entre o usuário e o \textit{chatbot}. Com isso, a Interação Humano-Computador (IHC) pretendida é influenciar positivamente na qualidade da interação, coletando \textit{feedback} do usuário (volume, repetição da resposta, etc.) e promovendo a personalização da experiência, onde o \textit{chatbot} se adapta às necessidades específicas e preferências individuais do usuário.
\label{fig:Prototipo}
\end{figure}

\end{frame}


\section{Considerações}
\begin{frame}[allowframebreaks]{Considerações}
    Em relação aos trabalhos relacionados, pode-se registrar que o trabalho \textit{Text-Generation-WebUI} \cite{text-generation-webui} foi essencial para pesquisa e entendimento do que poderia ser feito com LLMs e todos os parâmetros que podem ser passados diretamente a uma LLM. Assim, como o uso inicial e a documentação da API do \textit{Text-Generation-WebUI}, similar à API do próprio Chat GPT, o projeto Ollama também possibilita uma API compatível com a API da OpenAI do Chat GPT. No entanto, sem interface, funcionando como um serviço para carregar e utilizar LLMs Via sua API ou CLI.

\begin{itemize}
    \item Ao longo do texto, buscou-se abordar e exemplificar assuntos pertinentes ao Processamento de Linguagem Natural (PLN), Interação Humano-Computador (IHC), \textit{chatbots} e modelos pré-treinados de Modelos de Linguagem de Grande Escala (LLMs), que contêm um conjunto de ``conhecimentos'' ou textos em diversos contextos. Além disso, foram apresentados trabalhos relacionados que utilizam IHC via \textit{prompt} e modelos pré-treinados (\textit{online} ou \textit{offline}).

    \item A proposta tem a intenção de gerar uma interação falada via \textit{prompt} (ou que permita também a interação digitada em chat) e de forma \textit{offline}, com um modelo baixado, instalado, configurado e adaptado para um contexto específico (a definir na próxima etapa).
    \item O tema é bastante recente, assim como as tecnologias que lhe dão suporte. Os modelos GPT disponíveis estão rapidamente gerando impactos na sociedade, seja de forma positiva ou negativa. É um fato que os GPTs vão influenciar as sociedades, facilitando ou não, seja como uma poderosa ferramenta de trabalho ou como um mecanismo de manipulação e condução de opiniões.
    \item Assim sendo, este trabalho não apenas busca o projeto de um protótipo, mas também promove uma reflexão sobre os assuntos e suas tecnologias.
\end{itemize}

\end{frame}

\section{Referências}  
\begin{frame}[allowframebreaks]{Referências}
    \bibliographystyle{IEEEtran}
    \bibliography{references.bib}
\end{frame}


\begin{frame}
  \titlepage
\end{frame}

\end{document}
